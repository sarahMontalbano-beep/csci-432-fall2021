\documentclass{article}

\usepackage[margin=1in]{geometry}
\usepackage{amsthm}

\usepackage{algorithm}
\usepackage{algorithmic}
\renewcommand{\algorithmicrequire}{\textbf{Input:}}
\renewcommand{\algorithmicensure}{\textbf{Output:}}
\renewcommand{\algorithmiccomment}[2]{\hspace{#1}$\triangleright$ {#2} \hfill }

\usepackage{amsfonts}
\def\N{{\mathbb N}}
\def\R{{\mathbb R}}
\def\Z{{\mathbb Z}}

\newcommand{\algref}[1]{Algorithm~\ref{alg:#1}}

\newcommand{\answer}{\textbf{Answer:}\vspace{1.8in}}

\title{Recursion Invariants}
\author{CSCI 432, Fall 2021}

\begin{document}
\maketitle

\section*{Recursion Invariants: Proving Partial Correctness}

Recall the\textsc{Hanoi} algorithm (given in \algref{hanoi});
see~\cite[Ch.~1]{textbook}.

\begin{algorithm}[h!]
    \caption{$\textsc{Hanoi}(n,src,dest,tmp)$}\label{alg:hanoi}
    \begin{algorithmic}[1]
        \REQUIRE $n \in \N$, and three towers with disks: $src,dest,tmp$ such
        that $P$
        \ENSURE $R$ (see below)
        %
        \IF{ $n>0$ }\label{algln:certificate:ifdirs}
            \STATE $\textsc{Hanoi}(n-1,src,tmp,dest)$
            \STATE move top disk from $src$ to $dest$
            \STATE $\textsc{Hanoi}(n-1,tmp,dest,src)$
        \ENDIF
    \end{algorithmic}
\end{algorithm}

\begin{enumerate}
    \item Suppose we have $N$ disks total. What are the assumptions on the input
        to the initial call $\textsc{Hanoi}(N,src,dest,tmp)$?

        \answer

    \item What does it mean for $\textsc{Hanoi}(N,src,dest,tmp)$ to execute
        correctly? What does it return / what does it need to accomplish?

        \answer

        Going forward, we call this statement $Q$.

    \pagebreak
    \item For a general call to the recursive algorithm what are the assumptions on the input
        (For convenience, suppose the call is: $\textsc{Hanoi}(n,A,B,C)$).

        \answer

        Going forward, we will call these assumptions $P$.

    \item What is the recursion invariant?

        The recursion invariant is the (compound) statement $R$ that can fill in
        the blank of the following sentence:
        \emph{At each recursive call $\textsc{Hanoi}(n,A,B,C)$,
        $R$ is satisfied (i.e., true).}  Moreover, $R$ is such that we can use
        it to prove INITIALIZATIOn, MAINTENCE, and END.

        For $\textsc{Hanoi}$, the recursion invariant $R$ is:
        \begin{itemize}
            \item There are currently no violations of smaller disks on larger
                disks. (Note: the world would crumble if this were violated at
                any time), AND
            \item the $n$ smallest disks are now on $B$. (Note: they began on
                $A$), AND
            \item no other disks have moved.
        \end{itemize}

    \item INITIALIZTION This is like the base case for recursion.  Colloquially,
        we ask ``Why is this
        true for the smallest input?'' (And, what is those inputs that
        would allow us to return without a recursive call?)  More formally, we
        can say: If $n=n_0,A,B,C$ satisfy $P$, then after the call to
        $\textsc{Hanoi}(n_0,A,B,C)$, the recursion invariant $R$ is satisfied.

        \answer

        Note: sometimes, just as in induction, there may be more than one base case.

    \pagebreak
    \item MAINTENCE: This part is JUST like induction.
        Let $k \in \N$ such that $k \geq n_0$.  Assume that, for all $n \in \N$ such that
        $n_0 \leq n \leq k$, the recursion invariant $R$ holds after a call to
        $\textsc{Hanoi}(n,*,*,*)$. (That was the equivalent to the inductive
        assumption).  Now, we must prove that $R$ holds after
        $\textsc{Hanoi}(k+1,A,B,C)$. (Hint: use the line numbers as you walk
        through the algorithm).

        \answer

    \item END: This is where we diverge from induction.  Since algorithms are
        finite, we can't go on forever. Colloquially, we say ``if the initial call
        $\textsc{Hanoi}(N,src,dest,tmp)$ finishes executing, then all $N$ disks
        (which were initially on $src$) have moved to $dst$.'' More formally, we
        can phrase this as: If the recursion invariant holds and the algorithm
        completes execution, then the post-condition $Q$ is satisfied.

        \answer

\end{enumerate}

\bibliographystyle{acm}
\bibliography{biblio}

\end{document}
