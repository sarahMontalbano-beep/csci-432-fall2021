\documentclass{article}
\usepackage{../fasy-hw}
\usepackage{algorithim}
\usepackage{algpseudocode}

%% UPDATE these variables:
\renewcommand{\hwnum}{3}
\title{Advanced Algorithms, Homework \hwnum}
\author{Sarah Montalbano}
\collab{\todo{list your collaborators here}}
\date{due: Wednesday, 13 October 2021}

\begin{document}

\maketitle

This homework assignment should be
submitted as a single PDF file both to D2L and to Gradescope.

General homework expectations:
\begin{itemize}
    \item Homework should be typeset using LaTex.
    \item Answers should be in complete sentences and proofread.
    \item You will not plagiarize, nor will you share your written solutions
        with classmates.
    \item List collaborators at the start of each question using the
        \texttt{collab} command.
    \item Put your answers where the \texttt{todo} command currently is (and
        remove the \texttt{todo}, but not the word \texttt{Answer}).
    \item If you are asked to come up with an algorithm, you are
        expected to give an algorithm that beats the brute force (and, if possible, of
        optimal time complexity). With your algorithm, please provide the following:
        \begin{itemize}
            \item \emph{What}: A prose explanation of the problem and the algorithm,
                including a description of the input/output.
            \item \emph{How}: Describe how the algorithm works, including giving
                psuedocode for it.  Be sure to reference the pseudocode
                from within the prose explanation.
            \item \emph{How Fast}: Runtime, along with justification.  (Or, in the
                extreme, a proof of termination).
            \item \emph{Why}: Statement of the loop invariant for each loop, or
                recursion invariant for each recursive function.
        \end{itemize}
\end{itemize}

{\bf
This homework can be submitted as a group of size $n \geq 1$.
}

%%%%%%%%%%%%%%%%%%%%%%%%%%%%%%%%%%%%%%%%%%%%%%%%%%%%%%%%%%%%%%%%%%%%%%%%%%%%%%
\collab{}
\nextprob{Sugar Packet Game}

In Section 2.2 of the textbook, Dr.~Erickson describes a two-player game.
He states, "It's not hard to prove that
as long as there are tokens on the board, at least one player can legally
move.''  Prove this statement is correct.

\paragraph{Answer:}
\begin{proof}
Claim: If the number of total tokens on the board $n \geq 1$, then at least one player can legally move.

Base Case: $n = 1$. In the base case, there is only one token on the board. The token belongs to either player 1 or player 2, and the other player has no tokens on the board. This implies that the other player (e.g. the player without tokens on the board) has already won the game. The player with $n = 1$ tokens can legally move 1 space to the right (if player 1) or down (if player 2) but cannot jump a token of the other player because the other player has no tokens on the board. There is also no valid reason to pass because it is impossible for other tokens to block. Thus, whichever player with tokens $n =1$ is able to legally move; the claim is obviously true for $n = 1$.

The claim can be proved by contradiction for the remaining $n > 1$. Suppose that at any given turn neither player has a legal move to make. This assumption requires that both player 1 (horizontal) and player 2 (vertical) are blocked by two opponent pieces. However, this arrangement requires that either the vertical player have multiple tokens in the same column, or the horizontal player have multiple tokens in the same row, both of which are impossible. Therefore, if the number of total tokens on the board is $n \geq 1$, then at least one player can legally move, which was to be shown.

\end{proof}

%%%%%%%%%%%%%%%%%%%%%%%%%%%%%%%%%%%%%%%%%%%%%%%%%%%%%%%%%%%%%%%%%%%%%%%%%%%%%%
\collab{None}
\nextprob{Repeated pattern}

Chapter 3, Problem 26.  Only one algorithm is needed. Bonus points for a
solution that beats $O(n^5)$ though!

\begin{itemize}
            \item \emph{What}: The problem is to find the maximum-area rectangular pattern that appears more than once in a given bitmap. The input is a two-dimensional array of bits $M[1..n, 1..n]$ and the output is the area of the largest repeated rectangular pattern in M. The brute force algorithm is to manually compare two distinct positions, A and B, in the bitmap. The algorithm then calculates all rectangles with A as the top-left corner, then all rectangles with B as a top left corner. Each rectangle in A and B can have a length $l$ and breadth $b$, so those are two more for loops. Then it compares every bit of both rectangles ($l * b$), and if they match then there is a pattern of area $l*b$ that repeats, and if $l*b$ is greater than the previously-seen maximum area, the area is updated. The procedure is repeated for all pairs of A and B. 

The brute-force algorithm can be optimized into $\mathcal{O}(n^4)$ by performing memoization to store previously-calculated values and avoiding calculating unnecessary/unreachable values. This occurs in Compare, which takes two points in the image, stores whether one point matches the other, then looks up rectangles of $x_1, y_1$ and $x_2, y_2$.  

\item \emph{How}: 

\begin{algorithm}
\begin{algorithmic}
\Function{RectangleSearch}{M[1..n, 1..n]}
\State score $\leftarrow$ 0
\State maxArea $\leftarrow$ 0
\Comment{For each pair of points ($x_1, y_1$) and ($x_2, y_2$) in the bitmap, compare them. If the score of these points is greater than the current max, then replace MaxArea with score.}
\For {$x_1$ = n to 0}
\For {$x_2$ = n to 0}
\For {$y_1$ = n to 0}
\For {$y_2$ = n to 0}
\State score $\leftarrow$ Compare($x_1, x_2, y_1, y_2)

\If{score $>$ maxArea}
\State maxArea $\leftarrow$ score
\EndIf
\EndFor
\EndFor 
\EndFor
\EndFor
\State \Return maxArea
\EndFunction
\end{algorithmic}
\end{algorithm}

\begin{algorithm}
\begin{algorithmic}
\Function{Compare}{$x_1, x_2, y_1, y_2$}
\If{horizArr[$x_1, x_2, y_1, y_2$] $\neq$ NULL}
\State horizontal $\leftarrow$ horizArr[$x_1, x_2, y_1, y_2$]
\ElsIf{Match[$x_1, y_1$] == Match[$x_2, y_2$]}
\State horizontal $\leftarrow$ 1 + horizArr[$x_1 - 1, x_2 - 1, y_1, y_2]$
\Else
\State horizontal $\leftarrow$ 0
\EndIf

\If{vertArr[$x_1, x_2, y_1, y_2$] $\neq$ NULL}
\State vertical $\leftarrow$ vertArr[$x_1, x_2, y_1, y_2$]
\ElsIf{Match[$x_1, y_1$] == Match[$x_2, y_2$]}
\State vertical $\leftarrow$ 1 + vertArray[$x_1, x_2, y_1 - 1, y_2 -1]$
\Else
\State vertical $\leftarrow$ 0
\EndIf

\If{scoreArr $\neq$ NULL}
\State score $\leftarrow$ scoreArr[$x_1, x_2, y_1, y_2$]
\ElsIf{vertical == 1 + vertArr[$x_1 - 1, x_2 - 1$] AND horizontal == 1 + horizArr[$x_1 -1, x_2 -1]$ AND Match[$x_1, y_1$] == Match[$x_2, y_2$]}
\State score $\leftarrow$ 1 + vertical + horizontal + Compare[$x_1 -1, x_2 -1, y_1 -1, y_2 -1$]
\ElsIf{vertical == 1 + vertArr[$y_1 - 1, y_2 - 1$] AND Match[$x_1, y_1$] == Match[$x_2, y_2$]}
\State score $\leftarrow$ 1 + vertical + Compare[$x_1 -1, x_2 -1, y_1 -1, y_2 -1$]
\ElsIf{horizontal == 1 + horizArr[$x_1 -1, x_2 -1$] AND Match[$x_1, y_1$] == Match[$x_2, y_2$]}
\State score $\leftarrow$ 1 + horizontal + Compare[$x_1 - 1, x_2 - 1, y_1 - 1, y_2 -1$]
\ElsIf{Match[$x_1, y_1$] == Match[$x_2, y_2$]}
\State score $\leftarrow$ 1
\Else
\State score $\leftarrow$ 0
\EndIf
\State horizArr[$x_1, x_2, y_1, y_2$] $\leftarrow$ horizontal
\State vertArr[$x_1, x_2, y_1, y_2$] $\leftarrow$ vertical
\State scoreArr[$x_1, x_2, y_1, y_2$] $\leftarrow$ score
\State \Return score
\EndFunction
\end{algorithmic}
\end{algorithm}

            \item \emph{How Fast}: There are $n^2$ possible rectangles with any given point as their top-left corner and $n^2$ points in each rectangle that must be compared. Because there are two rectangles being compared at any point, the $\mathcal{O}(n^4)$ comparisons for each rectangle is doubled for an overall complexity of $\mathcal{O}(n^8).$ However, this is optimized into a final runtime of $\mathcal{O}(n^4)$ by memoization within the Compare function that eliminates repeated calculations and unnecessary calculations. With memoization, compare actually runs in constant time, which cuts down $\mathcal{O}(n^4)$.  

            \item \emph{Why}: I really still don't understand loop invariants, but my guess is, for each of the outer for-loops in RectangleSearch, that maxArea holds the maximum score for any combination of checked $x_1, x_2, y_1$, and $y_2$ values. 
        \end{itemize}

Consulted \url{https://stackoverflow.com/questions/56701990/bitmap-pattern-recogination} for prose explanation and description of optimization technique.

%%%%%%%%%%%%%%%%%%%%%%%%%%%%%%%%%%%%%%%%%%%%%%%%%%%%%%%%%%%%%%%%%%%%%%%%%%%%%%
\collab{None}
\nextprob{Greedy Algorithm}

What is the loop invariant of the for loop for the \textsc{GreedySchedule}
algorithm in Section 4.2 of the textbook?  Please state all relevant statements
(post-condition, pre-condition, loop invariant, and loop guard).  It may be
helpful to rewrite this for loop as a while loop and work with that.  Prove that
your loop invariant is correct.

\paragraph{Answer: }

The GreedySchedule algorithm always adds in the candidate class with the earliest finish time.

Loop Guard, G: $i \leq n$

Pre-condition, P: The input, $Y$, is a set of non-overlapping classes, sorted in increasing order by finishing time.

Post-condition, Q: The output, $X$, is a set of non-overlapping classes. (Doesn't necessarily need to be the optimal set of classes, only non-overlapping and therefore a viable schedule). 

Loop Invariant, $L_i$: The solution $X$ produced by GreedySchedule from $1$ to $i$  is a subset of some optimal solution, Opt, and is therefore non-overlapping.

\begin{proof}
Initialization, prove $\mathcal{P} \Rightarrow L_1$: At the initialization step, no classes have been added to the solution set $X$, so $X$ is trivially sorted and trivially non-overlapping. $L_1$ is implied from the precondition. 

Maintenance, prove $\mathcal{G} \wedge L_i \Rightarrow L_{i+1}$: Assume we are within our loop guard ($i \leq n$) and $L_i$ is true for some $i$. Let Opt be some solution set that includes the $i$ jobs scheduled so far. Let $j$ be the next class the algorithm schedules; if $j$ is in Opt, then that is. However, if $j$ is not within Opt, then let $j'$ be the class in Opt with the earliest finish time not scheduled in the $i$ classes scheduled so far. $j'$ must be a candidate since it doesn't conflict with any other class; therefore, the finish time of $j'$ must be greater than or equal to the finish time of $j'$. Thus we can remove $j'$ from Opt and add $j$, since $j$ does not conflict. 

End, prove $\neg G \wedge L \Rightarrow Q$: We assume now that $i > n$, which ends our loop. Our invariant is true throughout, and as such the solution produced in the final iteration is trivially a subset of itself. Therefore, the postcondition is true and the final solution is a set of non-overlapping classes. 
\end{proof}

Consulted \url{https://home.ttic.edu/~avrim/Algo19/lectures/greedy1.pdf} for a similar scheduling problem.
\end{document}
