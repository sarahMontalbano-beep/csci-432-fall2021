\documentclass{article}
\usepackage{../fasy-hw}

%% UPDATE these variables:
\renewcommand{\hwnum}{3}
\title{Advanced Algorithms, Homework \hwnum}
\author{Sarah Montalbano}
\collab{\todo{list your collaborators here}}
\date{due: Wednesday, 13 October 2021}

\begin{document}

\maketitle

This homework assignment should be
submitted as a single PDF file both to D2L and to Gradescope.

General homework expectations:
\begin{itemize}
    \item Homework should be typeset using LaTex.
    \item Answers should be in complete sentences and proofread.
    \item You will not plagiarize, nor will you share your written solutions
        with classmates.
    \item List collaborators at the start of each question using the
        \texttt{collab} command.
    \item Put your answers where the \texttt{todo} command currently is (and
        remove the \texttt{todo}, but not the word \texttt{Answer}).
    \item If you are asked to come up with an algorithm, you are
        expected to give an algorithm that beats the brute force (and, if possible, of
        optimal time complexity). With your algorithm, please provide the following:
        \begin{itemize}
            \item \emph{What}: A prose explanation of the problem and the algorithm,
                including a description of the input/output.
            \item \emph{How}: Describe how the algorithm works, including giving
                psuedocode for it.  Be sure to reference the pseudocode
                from within the prose explanation.
            \item \emph{How Fast}: Runtime, along with justification.  (Or, in the
                extreme, a proof of termination).
            \item \emph{Why}: Statement of the loop invariant for each loop, or
                recursion invariant for each recursive function.
        \end{itemize}
\end{itemize}

{\bf
This homework can be submitted as a group of size $n \geq 1$.
}

%%%%%%%%%%%%%%%%%%%%%%%%%%%%%%%%%%%%%%%%%%%%%%%%%%%%%%%%%%%%%%%%%%%%%%%%%%%%%%
\collab{}
\nextprob{Sugar Packet Game}

In Section 2.2 of the textbook, Dr.~Erickson describes a two-player game.
He states, "It's not hard to prove that
as long as there are tokens on the board, at least one player can legally
move.''  Prove this statement is correct.

\paragraph{Answer:}
\begin{proof}
Claim: If the number of total tokens on the board $n \geq 1$, then at least one player can legally move and either player 1 (horizontal) or player 2 (vertical) wins eventually.

Base Case: $n = 1$. In the base case, there is only one token on the board. The token belongs to either player 1 or player 2, and the other player has no tokens on the board. This implies that the other player (e.g. the player without tokens on the board) has already won the game. The player with $n = 1$ tokens can legally move 1 space to the right (if player 1) or down (if player 2) but cannot jump a token of the other player because the other player has no tokens on the board. Thus, whichever player with tokens $n =1$ is able to legally move; the claim is true for $n = 1$.

Inductive Assumption: Assume that, for all $k > 1$ and $k = n$, that at least one player can legally move and either player 1 or player 2 wins eventually. 

Inductive Step: If for all tokens $k > 1$ and $k = n$ at least one player can legally move, and either player 1 or player 2 wins eventually, then the claim must be true for $k + 1$ as well. The total number of tokens may not exceed $2n$, where $n$ is the dimension of the game board, and each player may have a maximum of $n$ tokens left in play. Each player, when there are a total of $k + 1$ tokens on the board, may have between 1 and $(k + 1)/2$ tokens on the board. The other player must have the reciprocal number of tokens. Thus if each player has $(k+1)/2$ tokens, each player has two legal move choices: the player may move right (if player 1) or down (if player 2), or may possibly be able to jump an opponent token. 
\end{proof}

%%%%%%%%%%%%%%%%%%%%%%%%%%%%%%%%%%%%%%%%%%%%%%%%%%%%%%%%%%%%%%%%%%%%%%%%%%%%%%
\collab{\todo{}}
\nextprob{Repeated pattern}

Chapter 3, Problem 26.  Only one algorithm is needed. Bonus points for a
solution that beats $O(n^5)$ though!

The brute force approach is to manually compare two distinct positions, A and B, in the bitmap. The algorithm then calculates all rectangles with A as the top-left corner, then all rectangles with B as a top left corner. Each rectangle in A and B can have a length $l$ and breadth $b$, so those are two more for loops. Then it compares every bit of both rectangles ($l * b$), and if they match then there is a pattern of area $l*b$ that repeats, and if $l*b$ is greater than the previously-seen maximum area, the area is updated. The procedure is repeated for all pairs of A and B. 

Time complexity will be on the order of $\mathcal{O}(n^8)$. There are $n^2$ possible rectangles with any given point as their top-left corner and $n^2$ points in each rectangle that must be compared. Because there are two rectangles being compared at any point, the $\mathcal{O}(n^4)$ comparisons for each rectangle is doubled for an overall complexity of $\mathcal{O}(n^8).$

The brute-force algorithm can be optimized into $\mathcal{O}(n^4)$ by performing the following memoization technique and storing previously-calculated comparisons. 

https://stackoverflow.com/questions/56701990/bitmap-pattern-recogination


%%%%%%%%%%%%%%%%%%%%%%%%%%%%%%%%%%%%%%%%%%%%%%%%%%%%%%%%%%%%%%%%%%%%%%%%%%%%%%
\collab{\todo{}}
\nextprob{Greedy Algorithm}

What is the loop invariant of the for loop for the \textsc{GreedySchedule}
algorithm in Section 4.2 of the textbook?  Please state all relevant statements
(post-condition, pre-condition, loop invariant, and loop guard).  It may be
helpful to rewrite this for loop as a while loop and work with that.  Prove that
your loop invariant is correct.

\paragraph{Answer: }

The GreedySchedule algorithm always adds in the candidate class with the earliest finish time.

Loop Guard, G:
Pre-condition, P: 
Post-condition, Q: 
Loop Invariant, $L_i$: The solution produced by GreedySchedule so far is a subset of the jobs scheduled in some optimal solution.

\end{document}
