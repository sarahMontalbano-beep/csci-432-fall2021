% csci432, N+1 st homework
% Due LDOC

\documentclass{article}
\usepackage{../fasy-hw}

%% UPDATE these variables:
\renewcommand{\hwnum}{n+1}
\title{Advanced Algorithms, Homework \hwnum}
\author{Sarah Montalbano}
\collab{n/a}
\date{due: 12/15/2021}


\begin{document}

\section*{$(n+1)$-st Homework Assignment}

{\bf{Due Date}}: Day of final exam meeting period.\\

This is your final assignment for CSCI 432.

Write a two-page paper describing to me how you have grown as a student,
computer scientist, mathematician, engineer, or a researcher in this class, and
more generally, in this semester.  To support your argument, you should include
your homework or writing samples (or excerpts from them) in an appendix as
evidence (and reference them!)

If you do not feel that you've grown, explain why.

Remember, style counts. Use complete sentences.

\newpage


In the depths of difficult homework problems, I often asked myself why I had taken Algorithms at all, as it was not strictly necessary for my degree, and less necessary still for my career. I resolved to do well, however, because I wanted to prove to myself that understanding the logic of algorithms was not out of my grasp. I have begun to learn to recognize patterns because of this course, while previously algorithms seemed disconnected and every problem seemed to require a brand-new approach. Further, I thoroughly enjoyed communicating algorithms concepts in several homework problems as well as in the final project, and I believe that my journalistic skills grew through those exercises. In general, this semester has seen the beginning of my career blossom and begun to endow me with the technical and statistical skills that are invaluable, but somewhat rare, in public policy research. 

Some context about me is necessary. I entered college intending to be a molecular biologist, which I had been obsessed with since I was 12. I switched to computer science during my first semester because I wanted a major that would give me options and I couldn't commit to a career in biology. I have worked in public policy part-time throughout my time at MSU, and I have received an offer to do education policy research about my home state of Alaska. All of this shows: 1) I am not a particularly talented computer scientist, and acquiring these skills is hard for me, 2) I am not likely to use computer science outside of simple data-science contexts, and 3) I don't particularly like computer science. 

I simply chose the major that stretched my abilities the farthest, and within this major, your class stretched me the most. What is remarkable is that I've found myself enjoying doing research into the problems, wrestling with the proof, and showing my thinking. I think that the difference is plain between my first and my second exams — I was too timid to attempt the induction problem on the first exam because I was not confident I could do it. On the second exam, however, I dove into designing an algorithm on the spot (despite not developing a dynamic programming solution). Thinking on my feet is an extremely important skill for radio and TV interviews, and your exams, as well as the presentation Q\&A period, have contributed to my readiness.

The most valuable skill that Algorithms has sharpened is pattern recognition. I dropped 232 the first time I attempted it out of the sheer frustration that each algorithm seemed to be a disconnected series of facts, useful for a specific problem, and I couldn't memorize it! For instance, it was only in this class that I realized that Dijkstra's algorithm and Prim's algorithm are almost identical, but slightly tailored for different problems (shortest path vs. minimum spanning tree). I have begun to realize that most problems need a tweaked solution based on some existing algorithm, rather than inventing a new concept. This realization has decreased my level of frustration and increased my satisfaction with computer science immensely. 

The problems I most enjoyed required me to explain an algorithms concept. 

\end{document}
