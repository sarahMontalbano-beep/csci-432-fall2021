\documentclass{article}
\usepackage{../fasy-hw}

%% UPDATE these variables:
\renewcommand{\hwnum}{1}
\title{Advanced Algorithms, Homework \hwnum}
\author{Sarah Montalbano}
\collab{none}
\date{due: 29 September 2021}

\begin{document}

\maketitle

This homework assignment should be
submitted as a single PDF file both to D2L and to Gradescope.

General homework expectations:
\begin{itemize}
    \item Homework should be typeset using LaTex.
    \item Answers should be in complete sentences and proofread.
    \item You will not plagiarize, nor will you share your written solutions
        with classmates.
    \item List collaborators at the start of each question using the
        \texttt{collab} command.
    \item Put your answers where the \texttt{todo} command currently is (and
        remove the \texttt{todo}, but not the word \texttt{Answer}).
\end{itemize}



%%%%%%%%%%%%%%%%%%%%%%%%%%%%%%%%%%%%%%%%%%%%%%%%%%%%%%%%%%%%%%%%%%%%%%%%%%%%%%
\collab{none}
\nextprob{Groups}

Chapter 2, Problem 4. For each of the recursive definitions, give the recurrence
relation and state the recursion invariant. For Part (a), prove that your
recursion invariant holds.  For Part (b), prove that your recursion terminates
using a decrementing function.

\paragraph{Answer}{Part (a): Give the recurrence relation, state the recursion invariants, and prove that the recursion invariant holds.

\begin{equation}
lcs(A, B, m, n) = 
\begin{cases}
    0 & \text{if m = 0 or n = 0}\\
    1 + lcs(A, B, m - 1, n -1) & \text{if $m, n > 0$ and $A[m] = B[n]$}.\\
    \max{lcs(A, B, m, n-1), lcs(A, B, m-1, n)}, & \text{otherwise}.
  \end{cases}
\end{equation}}

The recursion invariant described here assumes that the values are being stored in a memoization table of dimensions len(A) + 1 and len(B) + 1. The longest common subsequence should thus be in the bottom-right corner position of the array (dimensions m,n). 

The first invariant: rows 0..m-1 of the memoization array are filled correctly

The second invariant: within row m of the memoization array (inner loop), columns 0..n-1 are filled correctly. 

We prove our loop invariants as follows. If this makes no sense, I'm sincerely apologize. 

Initialization: When the algorithm enters the outer loop (for loop from 1 to $< len(A)$) for the first time, we calculate the right value of lcs(1, 0) = 0 by the base case of our recurrence relation (if m or n = 0). Similarly, as we enter the nested loop (for loop from 1 to len(B)), the right value for lcs(0, 1) = 0 is calculated by the base case of our recurrence relation (if m or n = 0). When both m and n are 1, we get lcs(1,1) which puts us in case 2 of our recurrence relation s.t. 1 + lcs(0,0) = 1. Thus the recurrence returns the correct values for lcs(1, 0), lcs(0,1) and lcs(1,1). 

Maintenance: Assuming the loop invariant is true for the k and p iteration of the algorithm, then the loop invariant is true for the k + 1 and p + 1 iteration of the algorithm. Let some $1 < k < len(A)$ and some $1 < p < len(B)$ correspond to a kth and pth iteration of the outer and inner loops. So if $k \neq p$ then LCS(k + 1, p + 1) = $\max{lcs(k + 1, p - 1 + 1), lcs(k - 1 + 1, p + 1)}$ =  $\max{lcs(k + 1, p), lcs(k, p + 1)}.$ And since we know that the loop invariant holds true for k and p, we can be certain that the recurrence returns correct values when $k \neq p$. (The recurrence is simply extrapolating out an additional kth or pth place in the memoization table based on either p or k and taking the maximum value).  

The correctness is more obvious when we consider when $k = p$. Then for k + 1 and p + 1 we have case 2 in the recurrence relation, where 1 + lcs(k - 1 + 1, p - 1 + 1) = 1 + lcs(k, p). Because we know lcs(k, p) has been computed correctly, then we can be certain that the recurrence returns correct values for k + 1 and p + 1.  

Termination: The loop ends when the index at the len(A) and len(B) has been calculated (and incidentally is the length of the longest common subsequence). Because all preceding indexes in the array have been calculated correctly (based on the maintenance step) we can be sure that the final call has the information it needs to calculate correctly (case 2 where len(A) and len(B) are equal) $lcs(m - 1, n - 1)$ or (case 3) $\max{lcs(m, n - 1), LCS(m - 1, n)}$. We can conclude because the initialization, maintenance, and termination steps are true that lcs() is correct. $\qedsymbol$

Consulted \url{https://www.cs.scranton.edu/~mccloske/courses/cmps240/longest_common_subseq_240.html} for clarification about the recurrence relation, invariants, and the Java code that was helpful to stare at for the recursion invariant proof. Also consulted \url{https://www.cs.dartmouth.edu/~deepc/Courses/S19/lecs/lec8.pdf} for a different take on the recursion invariant (I don't really understand the symbols here). 

\newpage
\paragraph{Answer}{Part (b): Give the recurrence relation, state the recursion invariants, and prove that the recurrence relation terminates using a decrementing function.}

The recurrence relation: 
\begin{equation}
SCS(A, B, m, n) = 
\begin{cases}
   $m + n$ & \text{$m = 0$ or $n = 0$}\\
    SCS(A, B, m-1, n-1) + 1 & \text{if m, n $>$ 0 and A[m] = B[n]}\\
    min(SCS(A, B, m, n - 1), SCS(A, B, m - 1, n))+ 1 & \text{otherwise}.
  \end{cases}
\end{equation}

The recursion invariant: The recursion invariants for populating the table are the same as the recursion invariant for the longest common subsequence problem. The first invariant is rows 0..m-1 of the memoization array are filled correctly; within row m of the memoization array (inner loop), columns 0..n-1 are filled correctly. But instead of selecting the maximum of SCS(m, n-1) and SCS(m - 1, n), we select the minimum. 

Proof of Termination: The shortest common supersequence decrements from the length of A (m) and the length of B (n) to 0. $m$ to $0$ and $n$ to $0$ are both well-ordered sets because every subset of m down to 0 has a least element. Then we can write a decrementing function that maps the state to these well-ordered sets. 

Thus we can show that $d : \mathcal{S} \rightarrow \mathbb{N}$. We must show that the $m$ and $n$ strictly decrease between each iteration of their respective loops. Looking at the recurrence relation itself, we must show that $m$ and $n$ decrease as we make each recursive call. And because $\mathbb{N}$ is a well-ordered set, we can thus conclude that S has a least element (0) and there must be an end to the recursion/looping. 

Let $d(*) =  m - i$, where $0 \leq i \leq m$ iterating down the outer loop and $d(*) = n - j$ where $0 \leq j \leq n$ iterating down the inner loop. Since $\mathbb{N}$ is well-ordered, $\min{image(d)}$ is realized, which means the algorithm terminates. $\qedsymbol$

(Not sure how to combine these two? Maybe $d(*) = (m - i) + (n - j)$?)

Consulted \url{https://simplecodehints.com/blog/shortest-common-supersequence-problem/} for prose explanation and confirmation of recurrence relation. 

This reference \url{https://www.techiedelight.com/shortest-common-supersequence-introduction-scs-length/} made me connect the dots for what a decrementing function would look like. 

\newpage
\paragraph{Answer}{Part (c): Recurrence relation and state recursion invariants.}

The longest increasing subsequence and longest decreasing subsequence algorithms are symmetric to each other. Let $i$ be the index of the location in array $A$ where the longest increasing subsequence ends and the longest decreasing subsequence starts. Let LIS() be the length of the longest increasing subsequence ending at A[j] and LDS() be the length of the longest decreasing subsequence starting at A[j]. Let $n$ be the length of A. 

So we must have a recurrence relation for the longest increasing subsequence: 
\begin{equation}
LIS(i) = 
\begin{cases}
1 + \max{LIS(j)} & \text{for all $ 0 < i < j$ and $A[i] < A[j]$}\\
\end{cases}
\end{equation}
and a recurrence relation for the longest decreasing subsequence:
\begin{equation}
LDS(i) = 
\begin{cases}
1 + \max{LDS(j)} & \text{for all $0 < j < i$ and $A[j] < A[i]$.}
\end{cases}
\end{equation}

The longest bitonic subsequence is found where $\max{LIS(i)+LDS(i)-1}$.

Consulted \url{https://www.geeksforgeeks.org/longest-bitonic-subsequence-dp-15/}  for prose explanation and connection to LIS and LDS. 

Recursion invariant: For the LIS() portion, our invariants are that for the outer loop $j$ = length of A down to 1 the memoization table fills rows len(A) to 1 correctly. The inner loop $i$ = $j -1$ down to 0 shows that for each row, the columns j-1 to 0 are filled correctly. Because LDS() is symmetric, our invariants are the same and the recursive calls simply select different values. 

\url{https://courses.engr.illinois.edu/cs374/sp2017/labs/solutions/lab7bis-sol.pdf} These solutions presented pseudocode for LIS and LDS next to each other and were helpful in figuring out the loop invariants. 

\paragraph{Answer}{Part (d): Recurrence relation and state recursion invariants}

The recurrence relation can again be broken into two parts, where Upper(i,j) denotes the length of the longest alternating subsequence of A[j..n] whose first element is larger than A[i] and whose second element is smaller than its first. Lower(i, j) denotes the length of the longest alternating subsequence of A[j..n] whose first element is smaller than A[i] and whose second element is larger than its first. 

The recurrences are thus: 

\begin{equation}
Upper(i,j) = 
\begin{cases}
0 & \text{if $j > n$}\\
Upper(i, j+1) & \text{if $j \leq n$ and $A[j] \leq A[i]$}\\
\max{Upper(i, j+1), 1 + Lower(j, j+1)} & \text{otherwise}
\end{cases}
\end{equation}

\begin{equation}
Lower(i,j) = 
\begin{cases}
0 & \text{if $j > n$}\\
Lower(i, j+1) & \text{if $j \leq n$ and $A[j] \geq A[i]$}\\
\max{Lower(i, j+1), 1 + Upper(j, j+1)} & \text{otherwise}
\end{cases}
\end{equation}

Then the length of the longest alternating subsequence is $\max{1 + Upper(j, j+1), 1 + Lower(j, j+1)}$.

Recursion Invariants: rows i starting from n decreasing to 1 of the memoization array are filled correctly. Within row i of the memoization array (inner loop), columns j-1 to 1 are filled correctly. 

Citation: \url{https://www.techiedelight.com/longest-alternating-subsequence/}

\paragraph{Answer}{Part (e): Recurrence relation and state recursion invariants}

Following the intuition of longest common subsequence and shortest common supersequence, I'm going to switch out the max for min. 

\begin{equation}
Upper(i,j) = 
\begin{cases}
0 & \text{if $j > n$}\\
Upper(i, j+1) & \text{if $j \leq n$ and $A[j] \leq A[i]$}\\
\min{Upper(i, j+1), 1 + Lower(j, j+1)} & \text{otherwise}
\end{cases}
\end{equation}

\begin{equation}
Lower(i,j) = 
\begin{cases}
0 & \text{if $j > n$}\\
Lower(i, j+1) & \text{if $j \leq n$ and $A[j] \geq A[i]$}\\
\min{Lower(i, j+1), 1 + Upper(j, j+1)} & \text{otherwise}
\end{cases}
\end{equation}

Then the length of the shortest alternating supersequence is $\min{1 + Upper(j, j+1), 1 + Lower(j, j+1)}$.

Recursion Invariant: rows i starting from n decreasing to 1 of the memoization array are filled correctly. Within row i of the memoization array (inner loop), columns j-1 to 1 are filled correctly. (Instead of selecting the max, we select min). 

\paragraph{Answer}{Part (f): Recurrence relation and state recursion invariants}
\begin{equation}
LXS(i,j) =
\begin{cases}
1 + \max{LXS(j, k)} & \text{$j < k \leq n$ and $A[i] + A[k] > 2A[j]$}
\end{cases}
\end{equation}
Recursion Invariant: For the outer loop i starting at $n-1$ decreasing to 1, rows n-1 to 1 are filled in correctly. For the second inner loop j from n down to i +1, the columns are filled in correctly. And for the innermost loop k from j + 1 to n, the calculation of LXS returns the right value.

Consulted \url{https://courses.engr.illinois.edu/cs374/sp2017/labs/solutions/lab7bis-sol.pdf} for pseudocode that was helpful to figuring out the loop invariants.
\end{document}
