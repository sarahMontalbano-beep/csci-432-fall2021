\documentclass{article}
\usepackage{../fasy-hw}

%% UPDATE these variables:
\renewcommand{\hwnum}{1}
\title{Advanced Algorithms, Homework \hwnum}
\author{Sarah Montalbano}
\collab{none}
\date{due: 27 September 2021}

\begin{document}

\maketitle

This homework assignment should be
submitted as a single PDF file both to D2L and to Gradescope.

General homework expectations:
\begin{itemize}
    \item Homework should be typeset using LaTex.
    \item Answers should be in complete sentences and proofread.
    \item You will not plagiarize, nor will you share your written solutions
        with classmates.
    \item List collaborators at the start of each question using the
        \texttt{collab} command.
    \item Put your answers where the \texttt{todo} command currently is (and
        remove the \texttt{todo}, but not the word \texttt{Answer}).
\end{itemize}



%%%%%%%%%%%%%%%%%%%%%%%%%%%%%%%%%%%%%%%%%%%%%%%%%%%%%%%%%%%%%%%%%%%%%%%%%%%%%%
\collab{none}
\nextprob{Groups}

Chapter 2, Problem 4. For each of the recursive definitions, give the recurrence
relation and state the recursion invariant. For Part (a), prove that your
recursion invariant holds.  For Part (b), prove that your recursion terminates
using a decrementing function.

\paragraph{Answer}{Part (a): Give the recurrence relation, state the recursion invariants, and prove that the recursion invariant holds.

\begin{equation}
lcs(A[1..m], B[1..n]) = 
\begin{cases}
    1 + lcs(A[2..m], B[2..n] & \text{if $A[1] = B[1]$}.\\
    max(lcs(A[1..m], B[2..n], lcs(A[2..m, B[1..n])), & \text{otherwise}.
  \end{cases}
\end{equation}}

The recursion invariant is: \todo{}

Proof: \todo{}

\paragraph{Answer}{Part (b): Give the recurrence relation, state the recursion invariants, and prove that the recurrence relation terminates using a decrementing function.}

The recurrence relation: 
\begin{equation}
SCS(A, B, m, n) = 
\begin{cases}
   $m + n$ & \text{$m = 0$}\\
   $m + n$ & \text{$n = 0$}\\
    SCS(A, B, m-1, n-1) + 1 & \text{A[m] = B[n]}\\
    min(SCS(A, B, m, n - 1), SCS(A, B, m - 1, n))+ 1 & \text{otherwise}.
  \end{cases}
\end{equation}

The recursion invariant: \todo{}

Proof of Termination: \todo{}

\paragraph{Answer}{Part (c): Recurrence relation and state recursion invariants.}

The longest increasing subsequence and longest decreasing subsequence algorithms are symmetric to each other. Let $i$ be the index of the location in array $A$ where the longest increasing subsequence ends and the longest decreasing subsequence starts. Let LIS() be the length of the longest increasing subsequence ending at A[j] and LDS() be the length of the longest decreasing subsequence starting at A[j]. Let $n$ be the length of A. 

So we must have a recurrence relation for the longest increasing subsequence: 
\begin{equation}
LIS(i) = 
\begin{cases}
1 + \max{LIS(j)} & \text{for all $ 0 < i < j$ and $A[i] < A[j]$}\\
\end{cases}
\end{equation}
and a recurrence relation for the longest decreasing subsequence:
\begin{equation}
LDS(i) = 
\begin{cases}
1 + \max{LDS(j)} & \text{for all $0 < j < i$ and $A[j] < A[i]$.}
\end{cases}
\end{equation}

The longest bitonic subsequence is found where $\max{LIS(i)+LDS(i)-1}$.

Consulted \url{https://www.geeksforgeeks.org/longest-bitonic-subsequence-dp-15/}  for prose explanation and connection to LIS and LDS. 

Recursion invariant: \todo{}

\paragraph{Answer}{Part (d): Recurrence relation and state recursion invariants}

The recurrence relation can again be broken into two parts, where Upper(i,j) denotes the length of the longest alternating subsequence of A[j..n] whose first element is larger than A[i] and whose second element is smaller than its first. Lower(i, j) denotes the length of the longest alternating subsequence of A[j..n] whose first element is smaller than A[i] and whose second element is larger than its first. 

The recurrences are thus: 

\begin{equation}
Upper(i,j) = 
\begin{cases}
0 & \text{if $j > n$}\\
Upper(i, j+1) & \text{if $j \leq n$ and $A[j] \leq A[i]$}\\
\max{Upper(i, j+1), 1 + Lower(j, j+1)} & \text{otherwise}
\end{cases}
\end{equation}

\begin{equation}
Lower(i,j) = 
\begin{cases}
0 & \text{if $j > n$}\\
Lower(i, j+1) & \text{if $j \leq n$ and $A[j] \geq A[i]$}\\
\max{Lower(i, j+1), 1 + Upper(j, j+1)} & \text{otherwise}
\end{cases}
\end{equation}

Then the length of the longest alternating subsequence is $\max{1 + Upper(j, j+1), 1 + Lower(j, j+1)}$.
\end{document}
